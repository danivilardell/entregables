\documentclass[12pt, a4papre]{article}
\usepackage[catalan]{babel}
\usepackage[unicode]{hyperref}
\usepackage{amsmath}
\usepackage{amssymb}
\usepackage{amsthm}
\usepackage{xifthen}
\usepackage{listings}
\usepackage{float}
\usepackage{siunitx}
\usepackage{graphicx}
\usepackage{indentfirst}

\newcommand{\norm}[1]{\lvert #1 \rvert}
\graphicspath{ {./Images/} }

\hypersetup{
    colorlinks = true,
    linkcolor = blue
}

\author{Daniel Vilardell}
\title{Previ practica 4 segona part}
\date{}

\begin{document}
	\maketitle
	
	\textbf{Qüestió 1:} Tenim que el periode es de $T = \frac{1}{4000} = 2.5\cdot 10^{-5}s$.
	
	\[
		\frac{T}{2} = 1.25\cdot 10^{-5} = R_9ln(2)330\cdot10^{-12} \implies R_9 = 54.6k\si{\ohm}
	\]	
	
	\textbf{Qüestió 2:}  Serveix com a desacoblament, per a derivar les tensions d'alterna a massa.\\
	
	\textbf{Qüestió 3:} 
	
	\[
		T_{OF} = \frac{10}{340} = 0.03s \implies PRF = \frac{1}{0.035} = 34Hz
	\] 
	
	\textbf{Qüestió 4:} Tenint en compte que $f = 40kHz$ tenim que
	
	\[
		\frac{10}{f} = 0.25ms
	\] 
	
	\textbf{Qüestió 5:} Tenim que
	
	\[
		T = 0.25 = ln(\frac{2V_{cc}-2}{V{cc}-2})R_{10}C \implies R_{10} = 1431 \si{\ohm}
	\]
	
	\textbf{Qüestió 6:} 
	
	\begin{itemize}
		\item $R_10 = 1431\si{\ohm}$
		\item $R_11 = 330k\si{\ohm}$
		\item $C_7 = 220nF$
		\item Periode: $T = C_7(R_{10}+0.693R_{11}) = 0.05062s$
		\item $PRF = \frac{1}{T} = 19.75Hz$
	\end{itemize}
	
\end{document}




