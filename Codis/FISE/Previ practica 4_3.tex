\documentclass[12pt, a4papre]{article}
\usepackage[catalan]{babel}
\usepackage[unicode]{hyperref}
\usepackage{amsmath}
\usepackage{amssymb}
\usepackage{amsthm}
\usepackage{xifthen}
\usepackage{listings}
\usepackage{float}
\usepackage{siunitx}
\usepackage{graphicx}
\usepackage{indentfirst}

\newcommand{\norm}[1]{\lvert #1 \rvert}
\graphicspath{ {./Images/} }

\hypersetup{
    colorlinks = true,
    linkcolor = blue
}

\author{Daniel Vilardell}
\title{Previ practica 4 tercera part}
\date{}

\begin{document}
	\maketitle
	
	\textbf{Qüestió 1:} Tenim que $1cm = 0.01m \implies T = \frac{0.02}{340} = 58.8ms$ i per tant tenim que
	
	\[
		f = \frac{1}{T} = \frac{340}{0.02} = 17kHz
	\]
	
	\textbf{Qüestió 2:}  En primer lloc tenim que
	
	\[
		V_c(\frac{T}{2}) = V_{cc} + (V_c(t_i)-V_{cc})e^{-\frac{\frac{T}{2}-t_i}{R_{12}C_{10}}}
	\]
	
	I per tant
	
	\[
		-\frac{V_{cc}}{2} = -\frac{3V_{cc}}{2}e^{\frac{\frac{T}{2}}{R_{12}C_{10}}} \implies \frac{1}{3} = e^{\frac{\frac{T}{2}}{R_{12}C_{10}}}
	\]
	
	\[
		T = 2R_{12}C_{10}ln(3) \implies f = \frac{1}{T} = \frac{1}{2R_{12}C_{10}ln(3)}
	\]
	
	\textbf{Qüestió 3:} 
	
	\[
		f_{min} = 15kHz = \frac{1}{2R_210^{-9}ln(3)} \implies R_{2_{max}} = 30.34k\si{\ohm}
	\]
	
	\[
		f_{max} = 20kHz = \frac{1}{2R_210^{-9}ln(3)} \implies R_{2_{max}} = 22.75k\si{\ohm}
	\]
	
	I per tant tenim que $22.75k\si{\ohm} < R_2 < 30.34k\si{\ohm}$.
	
\end{document}




