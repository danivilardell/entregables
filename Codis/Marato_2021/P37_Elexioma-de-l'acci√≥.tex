\documentclass[12pt, a4papre]{article}
\usepackage[catalan]{babel}
\usepackage[unicode]{hyperref}
\usepackage{amsmath}
\usepackage{amssymb}
\usepackage{amsthm}
\usepackage{xifthen}
\usepackage{listings}
\usepackage{float}
\usepackage{siunitx}
\usepackage{graphicx}
\usepackage{indentfirst}

\newcommand{\norm}[1]{\lvert #1 \rvert}
\graphicspath{ {./Images/} }

\hypersetup{
    colorlinks = true,
    linkcolor = blue
}

\author{Elexioma de l'acció}
\title{37. Regla de la cadena}
\date{}

\begin{document}
	\maketitle
	
	El pobre Felix haurà de buscarse una altra forma de ser més $M.D.L.R.$ ja que, encara que encara no ho sàpiga, aquesta cadena no pot existir.
	
	\begin{proof} Suposem que fos cert i arrivarem a una contradicció.
	
	Sigui $C$ la cadena de cardinal no numerable, i $C_i$  la partició dels naturals a la posició $i$ de la cadena. Considerem el ordre classic dels naturals, i apuntem que cualsevol subconjunt dels naturals te minim, ja que $\mathbb{N}$ es un conjunt ben ordenat.
	
	Considerem la successió $a_n$ definida de la seguent forma:
	
	\[
		a_i = min(C_{i+1}\backslash C_i)
	\]
	
	Esta ben definida ja que $C_{i+1}\backslash C_i \neq \emptyset$ al ser la inclussió estricte i ser $C_{i+1}$ i $C_{i}$ parts de $\mathbb{N}$. També existeix el mínim ja que $\mathbb{N}$ es un conjunt ben ordenat. A mes cal apuntar que $a_i \neq a_j \quad \forall i \neq j$ ja que els nombres dins de $C_{i+1}$ no estan repetits al ser un conjunt i per tant, com que $a_i \in C_{i+1}\backslash C_i \implies a_j \not\in C_{i+1}\backslash C_i$ ja que per la definició (considerant j > i sense perdua de generalitat) $C_i \in C_j \implies a_i \in C_j \implies a_i \not \in C_{j+1}\backslash C_j$.
	
	El conjunt format per els elements de la seqüencia serà de igual cardinal que la cadena $C$, i també serà un subconjunt dels naturals cosa que porta a contradicció, ja que els naturals son un conjunt numerable.
		
	\end{proof}
	
	
	
	
\end{document}