\documentclass[12pt, a4papre]{article}
\usepackage[catalan]{babel}
\usepackage[unicode]{hyperref}
\usepackage{amsmath}
\usepackage{amssymb}
\usepackage{amsthm}
\usepackage{xifthen}
\usepackage{listings}
\usepackage{float}
\usepackage{siunitx}
\usepackage{graphicx}
\usepackage{indentfirst}

\newcommand{\norm}[1]{\lvert #1 \rvert}
\graphicspath{ {./Images/} }

\hypersetup{
    colorlinks = true,
    linkcolor = blue
}

\author{Elexioma de l'acció}
\title{P84. H4H4H4H4}
\date{}

\begin{document}
	\maketitle
	
	\begin{proof} Com que aquest joc es impossible que quedi en taules (sempre perd algú), finalitza en un nombre finit de tirades i existeixen posicions guanyadores i perdedores podem afirmar que per cada m i n en particular o be el Max o l'Enric tindrà estrategia guanyadora. Veurem ara que sempre existeix estrategia guanyadora pel Max.
	
	Suposem que el Max no tingues raó, es a dir que estigues en una posició perdedora i per tant tots els moviments envien a posicions guanyadores. Aleshores considerem el cas en que el Max tira a la casella $(n, m)$. Qualsevol tirada que faci el seu rival la podria haver fet ell com a tirada inicial, i en concret, la tirada que porta a la posició perdedora. Per tant el Max tenia una tirada que portava a una posició perdedora i arribem a contradicció.
	
	Amb això però demostrem que la estrategia guanyadora pel Max existeix, que la tingui o no ja no es el seu problema.
	\end{proof}
	
\end{document}