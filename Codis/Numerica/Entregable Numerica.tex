\documentclass[12pt, a4papre]{article}
\usepackage[catalan]{babel}
\usepackage[unicode]{hyperref}
\usepackage{amsmath}
\usepackage{amssymb}
\usepackage{amsthm}
\usepackage{xifthen}

\newcommand{\norm}[1]{\lvert #1 \rvert}

\hypersetup{
    colorlinks = true,
    linkcolor = blue
}
\newtheorem*{theorem*}{Theorem}
\newtheorem*{lemma}{Prop}

\author{Daniel Vilardell}
\title{Entregable Numèrica}
\date{}

\begin{document}
	\maketitle
	\begin{lemma} La inversa de una matriu triangular $A$ es una matriu triangular. \end{lemma}
	\begin{proof}
		La demostració es farà per matrius triangulars inferiors, ja que si $A$ i $A^{-1}$ triangular inferior $\implies AA^{-1} = I \implies (AA^{-1})^T = I^T = I \implies (A^{-1})^TA^T = I$ i en aquest cas $B=A^T$ i $B^{-1} = (A^{-1})^T$ serien matrius triangulars superiors.
		
		Per tal de fer la demostració aplicarem operacions elementals a la matriu A per tal de transformarla en la identitat. Com be sabem de Algebra Lineal, aplicar operacions elementals a una matriu es el mateix que multiplicar per la matriu elemental E equivalent.
		
		En primer lloc cal mencionar que totes les operacions elementals que usarem seran les de restar files $i$ files $j$ amb $j < i$ ja que ens interessa eliminar tots els nombres de la part inferior de la matriu $A$. La matriu elemental que ens permet eliminar un nombre $a_{ij}$ es la identitat amb el coeficient $m = \frac{a_{ij}}{a_{ii}}$ a la posició $i, j$ de la matriu. 
		
		Sigui $E_i$ la matriu elemental de la iessima operació elemental que fem per a transformar $A$ amb la identitat. Aleshores:
		
		\[
			E_m \cdots E_1 \cdot A = I
		\]
		
		Per la raó comentada abans totes les matrius $E_i$ son triangulars inferiors ja que $A$ només te elements a la part inferior de la matriu i per tant $E_i$ també. També podem afirmar que $A^{-1} = E_m \cdots E_1$ i que com el producte de matrius triangulars inferiors es una matriu triangular inferior, $A^{-1}$ es triangular inferior.
	\end{proof}
\end{document}










