\documentclass[12pt, a4papre]{article}
\usepackage[catalan]{babel}
\usepackage[unicode]{hyperref}
\usepackage[dvipsnames]{xcolor}
\usepackage{amsmath}
\usepackage{amssymb}
\usepackage{amsthm}
\usepackage{xifthen}
\usepackage{siunitx}
\usepackage{xcolor}
\usepackage{float}
\usepackage{listings}
\usepackage{setspace}
\usepackage{graphicx}
\usepackage{tikz,lipsum,lmodern}
\usepackage[most]{tcolorbox}
\usepackage{fancyvrb}
\usepackage{circuitikz}
\usepackage{indentfirst}
\usepackage{verbatimbox}
\usepackage{verbatim}
\usepackage[utf8]{inputenc}
\definecolor{mygreen}{RGB}{28,172,0} % color values Red, Green, Blue
\definecolor{mylilas}{RGB}{170,55,241}
\graphicspath{ {./images/} }


\newcommand{\norm}[1]{\lvert #1 \rvert}

\hypersetup{
    colorlinks = true,
    linkcolor = blue
}

\author{}
\title{Implementacio de l’algorisme del simplex primal}
\date{}

\begin{document}
	\maketitle
	\begin{center}
		\begin{tabular}{ |c | c  c |}
			\hline
			\textbf{Nom} 		& \textbf{DNI}		& \textbf{Conjunt de dades} \\ \hline
			Ruben Aciego 		&--- 		 		& 1  			\\ 
			Daniel Vilardell 		&48109585W 		& 65\\
			\hline
		\end{tabular}
	\end{center}
	\tableofcontents
	
	\newpage
	
	\section{Breu explicació del funcionament del codi}
	
	La implementació que hem fet del simplex consta de dos fases, la fase I on trobem una SBF del PL donata partir d'afegir la matriu identitat a A i agafar com a VB inicials els corresponents al les columnes de la matriu afegida. Tot i que tenim les dos fases, fem tots els calculs dins de la mateixa funcio, anomenada iteració, a on li passem com a parametre un bool que marca si estem a la fase I o a la fase II.
	
	Per tal de detectar la infactibilitat, despres d'aplicar la fase I mirarem si el valor de la funció objectiu de la fase inicial es major que 0, ja que en aquest cas s'hauria trobat una SBF optima del problema de fase I que depen de VB afegides que no pertanyen al PL original. En el cas de que detecti la infactibilitat, el problema retornara -1.
	
	Per tal de detectar si el PL es il·limitat, a cada iteració de la fase II mirarem si totes les direccions basiques son majors o iguals a 0, i en el cas que es doni tal event, retornarem -1 de la funció de iteració i retornarem -2 de la funcio simplex. Les dades que es mostraran despres de informar que el problema es ilimitat seran la informació del ultim vertex que te una aressta amb direcció de descens negativa i $\theta^*$ infinita.
	
	\newpage
	\section{Solució dels problemes de Ruben Aciego}
	
	\subsection{PL1}
	
	\subsection{PL2}
	
	\subsection{PL3}
	
	\subsection{PL4}
	
	\newpage
	\section{Solució dels problemes de Daniel Vilardell}
	
	\subsection{PL1}

	{\scriptsize \VerbatimInput{./solucio/SolucioPL1_D.txt}}
	
	\subsection{PL2}

	{\scriptsize \VerbatimInput{./solucio/SolucioPL2_D.txt}}
	
	\subsection{PL3}
	
	{\scriptsize \VerbatimInput{./solucio/SolucioPL3_D.txt}}
	\newpage
	\subsection{PL4}
	
	{\scriptsize \VerbatimInput{./solucio/SolucioPL4_D.txt}}
	

	
\end{document}